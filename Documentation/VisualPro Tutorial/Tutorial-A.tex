\documentclass[10pt]{article}  

%%%%%%%% PREÁMBULO %%%%%%%%%%%%
\usepackage[english]{babel} %Indica que escribiermos en español
\usepackage[utf8]{inputenc} %Indica qué codificación se está usando ISO-8859-1(latin1)  o utf8  
\usepackage{amsmath} % Comandos extras para matemáticas (cajas para ecuaciones,
% etc)
\usepackage{amssymb} % Simbolos matematicos (por lo tanto)
\usepackage{graphicx} % Incluir imágenes en LaTeX
\usepackage{color} % Para colorear texto
\usepackage{subfigure} % subfiguras
\usepackage{ulem}
\usepackage{amsmath}%%Usar ecuaciones
%%%%%%%%%%%%%CODIGO%%%%%%%%%%%%%%%%%%%%%%
\usepackage{listings} %Sirve para pegar codigo fuente de programas
\usepackage{caption}
\DeclareCaptionFont{white}{\color{white}}
\DeclareCaptionFormat{listing}{%
  \parbox{\textwidth}{\colorbox{gray}{\parbox{\textwidth}{#1#2#3}}\vskip-4pt}}
\captionsetup[lstlisting]{format=listing,labelfont=white,textfont=white}
\lstset{frame=lrb,xleftmargin=\fboxsep,xrightmargin=-\fboxsep}
%%%%%%%%%%%%%CODIGO%%%%%%%%%%%%%%%%%%%%%%
\usepackage{float} %Podemos usar el especificador [H] en las figuras para que se
% queden donde queramos
\usepackage{capt-of} % Permite usar etiquetas fuera de elementos flotantes
% (etiquetas de figuras)
\usepackage{enumerate} % enumerados
\usepackage{sidecap} % Para poner el texto de las imágenes al lado
	\sidecaptionvpos{figure}{c} % Para que el texto se alinie al centro vertical
\usepackage{caption} % Para poder quitar numeracion de figuras
\usepackage{commath} % funcionalidades extras para diferenciales, integrales,
% etc (\od, \dif, etc)
\usepackage{cancel} % para cancelar expresiones (\cancelto{0}{x})
 
\usepackage{anysize} 					% Para personalizar el ancho de  los márgenes
\marginsize{2cm}{2cm}{2cm}{2cm} % Izquierda, derecha, arriba, abajo

\usepackage{appendix}

%Para la creación de tablas
\usepackage{array}
\newcolumntype{P}[1]{>{\centering\arraybackslash}p{#1}}
\newcolumntype{M}[1]{>{\centering\arraybackslash}m{#1}}
\usepackage{makecell}



%Para la creacion de cuadros de colores
\usepackage{tcolorbox}
\tcbuselibrary{listingsutf8}
% Cuadro numerado para ejemplos de cuadros de colores
\newtcolorbox[auto counter, number within=section]{example}[2][]
{colback=red!5!white,colframe=red!75!black,
fonttitle=\bfseries, title=Example~\thetcbcounter: #2,#1}

\newtcolorbox[auto counter, number within=section]{tip}[2][]
{colback=green!5!white,colframe=green!75!black,
fonttitle=\bfseries, title=Tip~\thetcbcounter: #2,#1}

% Para que las referencias sean hipervínculos a las figuras o ecuaciones y
% aparezcan en color
\usepackage[colorlinks=true,plainpages=true,citecolor=blue,linkcolor=blue]{hyperref}
%\usepackage{hyperref} 
% Para agregar encabezado y pie de página
\usepackage{fancyhdr} 
\pagestyle{fancy}
\fancyhf{}
\fancyhead[L]{\footnotesize VPLWVST} %encabezado izquierda
\fancyhead[R]{\footnotesize VP-TA}   % dereecha
\fancyfoot[R]{\footnotesize Tutorial A}  % Pie derecha
\fancyfoot[C]{\thepage}  % centro
\fancyfoot[L]{\footnotesize VisualPro: A Lightweight; Visual Scripting Tool}  %izquierda
\renewcommand{\footrulewidth}{0.4pt}


\usepackage{listings} % Para usar código fuente
\definecolor{dkgreen}{rgb}{0,0.6,0} % Definimos colores para usar en el código
\definecolor{gray}{rgb}{0.5,0.5,0.5} 
% configuración para el lenguaje que queramos utilizar
\lstset{language=Matlab,
   keywords={break,case,catch,continue,else,elseif,end,for,function,
      global,if,otherwise,persistent,return,switch,try,while},
   basicstyle=\ttfamily,
   keywordstyle=\color{blue},
   commentstyle=\color{red},
   stringstyle=\color{dkgreen},
   numberstyle=\tiny\color{gray},
   stepnumber=1,
   numbersep=10pt,
   backgroundcolor=\color{white},
   tabsize=4,
   showspaces=false,
   breaklines=true,
   showstringspaces=false}

\newcommand{\sen}{\operatorname{\sen}}	% Definimos el comando \sen para el seno
%en español
%%%%%%%% TERMINA PREÁMBULO %%%%%%%%%%%%

\begin{document}

%%%%%%%%%%%%%%%%%%%%%%%%%%%%%%%%%% PORTADA %%%%%%%%%%%%%%%%%%%%%%%%%%%%%%%%%%%%%%%%%%%%
																					%%%
\begin{center}																		%%%
\newcommand{\HRule}{\rule{\linewidth}{0.5mm}}									%%%\left
 																					%%%
\begin{minipage}{0.48\textwidth} \begin{flushleft}
\includegraphics[scale = 0.15]{Figures/Logo/UWTSD-Logo}
\end{flushleft}\end{minipage}
\begin{minipage}{0.48\textwidth} \begin{flushright}
\includegraphics[scale = 0.4]{Figures/Logo/VP-Logo-Large}
\end{flushright}\end{minipage}

													 								%%%
\vspace*{-1.5cm}								%%%
																					%%%	
\textsc{\huge Visual Pro\\ \vspace{5px} Tutorial A}\\[1.5cm]	

\textsc{\LARGE A beginners guide to Visual Scripting}\\[1.5cm]													%%%

\textsc{\Large VisualPro: A Lightweight; Visual Scripting Tool}\\[0.5cm]

    																				%%%
 			\vspace*{1cm}																		%%%
																					%%%
\HRule \\[0.4cm]																	%%%
{ \huge \bfseries Tutorial: Classes and Objects}\\[0.4cm]	%%%
 																					%%%
\HRule \\[1.5cm]																	%%%
 																				%%%
																					%%%
\begin{minipage}{0.4\textwidth}													%%%
\begin{flushleft} \large															
\textit{Authors:}\\
Edward Patch\\
Student Number: 1801492\\
\end{flushleft}																		%%%
\end{minipage}		
																%%%
\begin{minipage}{0.5\textwidth}		
\vspace{-0.6cm}											%%%
\begin{flushright} \large															%%%
\textit{Supervisor:} \\
Mike Dacey														%%%									
\end{flushright}																	%%%
\end{minipage}	
\vspace*{1cm}

\vspace{2cm} 																				
\begin{center}																					
{\large 16 February 2022}																	%%%
 			\end{center}												  						
\end{center}							 											
																					
\newpage																		
%%%%%%%%%%%%%%%%%%%% TERMINA PORTADA %%%%%%%%%%%%%%%%%%%%%%%%%%%%%%%%

\tableofcontents
\thispagestyle{fancy}

\newpage
\section{Learning Objectives}
The following learning objectives are as follows:
\begin{enumerate}
    \item Understanding the VisualPro application.
    \item Understanding the terminologies of classes, objects and methods.
    \item Writing several classes in a Visual Scripting environment.
\end{enumerate}


\newpage
\section{Introduction}
    \subsection{What is VisualPro?}
        VisualPro aims to create a lightweight Visual Scripting tool that encourages individuals to the development field. VisualPro enables users to create code structures to help develop ideas into reality. \\\textbf{Note:} VisualPro does not have logical programming tools in its current implementation.

    \subsection{What will the tutorial cover?}
        The tutorial covers VisualPro Environment of how to use the software, found in section~\ref{sec:vp-env}, page~\pageref{sec:vp-env} and terminology found in section~\ref{sec:terminology}, page~\pageref{sec:terminology}.
        The tutorial teaches the basics of Object-Oriented and how to implement the structure within VisualPro, found in section~\ref{sec:object-orientation}, page~\pageref{sec:object-orientation}.
        
        \begin{example}{Object-Oriented Languages (or) Languages with Object-Oriented Features}
            C++, C\# and Java. To view other object-oriented languages,\\~\href{https://worddisk.com/wiki/List_of_object-oriented_programming_languages/}{WordDisk - Languages with object-oriented features}~\cite{word_disk_list_2018}.
        \end{example}

\section{VisualPro Environment}
\label{sec:vp-env}
    \subsection{Features}
        VisualPro offers a few features such as:
        \begin{itemize}
            \item Classes, Functions and Arguments, and Variables.
            \item Saving in Multiple Languages.
            \item Drag and Drop Elements and Text Areas.
            \item Property Windows to Control Arguments and Relationships.
        \end{itemize}

    \subsection{Known Bugs}
        A couple of bugs include:-
        \begin{itemize}
            \item Arguments for functions are not currently available.
            \item Deleting containers does not mathematically reset the following location or move existing containers backwards.
        \end{itemize}

    \subsection{Saving Progress}
        After completing a tutorial, select the language and press save.

        \begin{tip}{Compatibility}
            As mentioned previously, not all languages support object-orientation. If a language is not compatible with a particular keyword, the code generator will ignore the selected syntax object.

            \textbf{Example Language:} C is not supported.
        \end{tip}

\section{Terminology}
\label{sec:terminology}
    \subsection{What terminologies exist within Object-Orintation programming?}

    \subsection{What is a class?}

    \subsection{What is a method?}

    \subsection{What is a object?}


\section{Object-Orientation}
\label{sec:object-orientation}

    \subsection{Exercise: Understanding the basics}

    \subsection{Exercise: Trying out a Class}

    \subsection{Exercise: Animal Types}

    \subsection{Exercise: Vehicle Components}

%%%%%%% References %%%%%%%%
\clearpage
\nocite{*}
\small{\bibliographystyle{IEEEtran}
\bibliography{ref}}
%%%%%%% References %%%%%%%%      
\end{document}